\chapter{CONCLUSION}
\label{chap-six}

\section*{Conclusion}
This chapter provides a conclusion to the work discussed in previous chapters.

\section{Summary}
This paper presents an investigation into Cellulose Acetate Tow (CA) as an acoustic absorption material. Research in this field is of growing interest, as noise pollution is considered to be a major health concern. Samples of CA were tested in various manufactured configurations in a Br�el \& Kj�r Impedance Tube Type 4206A. A variety of commercial absorption materials were also tested and are compared to the acoustic performance characteristics of Cellulose Acetate Tow.\\
\indent To appropriately analyze the problem, necessary mathematical background is developed and a review of current literature in the theory regarding acoustic absorption has been reviewed in Chapter \ref{chap-two}. Key parameters for the fibers, chemistry, and poroelasticity have been presented and discussed. Empirical models of porous materials and measurement techniques are also discussed. Finally, the impedance tube method definition as used by the experimental set-up is also covered.\\
\indent Chapter \ref{chap-three} covers the experiment design and configuration used in the investigation. A complete list of materials is presented. Considerations regarding the testing preparation are also discussed. The impedance tube testing process is given. A detailed step-by-step outline for the equipment can be found in Appendix \ref{append-B}.\\
\indent Experimental results are presented in Chapter \ref{chap-four} for each specimen. Results show data from $500 \text{Hz}$ to $6300 \text{Hz}$, however, tests on lower frequencies were also performed. The low frequency information have been omitted in order to show the most relevant data. Implications of the experimental data shown in Chapter \ref{chap-four} are discussed in detail in Chapter \ref{chap-five}. After the analysis of the data, the appropriate concerns and findings have been discussed.\\
\indent Similar experiments designed for measuring acoustic properties of biologically friendly materials have been performed in literature \cite{Ersoy2009} and \cite{Wassilieff1996}. This paper aims to provide a new material in the field of biologically friendly porous media used for acoustic absorption. As can be see in Chapters \ref{chap-four} and \ref{chap-five}, Cellulose Acetate shows promise as an effective alternative to current materials.\\
\indent While it can be seen in Figure \ref{fig:AfigSOCKScomparefiberglass} that some CA materials perform slightly better or slightly worse than fiberglass, it should be noted that cost of the materials is important. This is both in financial and logistical cost.The samples with the best performance, the high denier, are more difficult to work with and thus have a higher logistical cost than the lower denier lesser performing materials. The issue with the higher denier materials is they are messy to work with and this is not suitable for all environments. A finer material is much easier to work with and while less performing, potentially more suitable.

\subsection{Future Work}
This work provides background for future research in this subject. Due to the complex nature of porous media absorption problems, the key poroelastic parameters needed for performance optimization are difficult to identify. Possible future work includes investigation in the optimal performance of the material. This can be done by considering a study on the ratio of filler material in the Cellulose Acetate. This was considered outside of the scope of the present project. Another potential research topic is investigating the accuracy of empirical methods discussed in Chapter \ref{chap-two} by comparing results to experimental data presented in this paper.
